\documentclass{scrartcl}
\usepackage[T1]{fontenc}
\usepackage{baskervald}
\usepackage{graphicx}
\usepackage{datetime}
\usepackage[]{minted}
\usepackage{tcolorbox}
\usepackage{etoolbox}
\BeforeBeginEnvironment{minted}{\begin{tcolorbox}[arc=0mm,colback=yellow!9!white,colframe=yellow!9!white]}
\AfterEndEnvironment{minted}{\end{tcolorbox}}
\AtBeginDocument{\renewcommand\contentsname{\textit{Table of Contents}}}
\title{Minimum Meaningful Product}
\author{Boris Kourtoukov}
\date{}
\begin{document}

\maketitle
\vspace{1cm}
\tableofcontents
\newpage

\begin{abstract}
\noindent
In his Distance essay Francisco Inchauste focuses on meaningful innovation and the concept of a Minimum Meaningful Product.
\end{abstract}

\section{Technium}

Technium is a term that is introduced early and referenced throughout Inchauste's essay. It refers to \textit{the Technium}, a term coined by Kevin Kelly.

\begin{quote}
	The Technium is an ecosystem with a behavior of its own, larger than the sum of modern art, social institutions and intellectual concepts.\cite{Herzog}
\end{quote}

This concept is described as our current reality, one where we use tools that we could never build ourselves. Tools that require the combined effort of hundreds or thousands in order to produce or understand. 

\newthought{With} this in tow the essay narrows it's focus on the burgeoning startup culture as it navigates the current state of the Technium. 

%\marginnote{test}

\section{Cultures of Innovation}

Inchauste argues that today's startups are too focused on the value of innovation for the sake of profit. Products are created with the \textit{perception} of radical innovation rather than \textit{actual} radical innovation. This often also comes with only meager improvements or benefits to the user, but with a highly over-extended attempt to define what they need in their lives.

A purely profit and growth success model is the driving force behind this. This model is ignorant of the many other factors that influence business success, i.e. creating distinct value for customers and having a clear meaning to the work or product produced.

\newthought{Inchauste points out} that money is only a single factor, not the whole picture. He argues that the often closed circuit and internal natures of startup funding and capital raising is at the heart of the trap many new companies fall in-to. More often than not accelerators are propelling teams for their talent, potential or direct skill set rather than the ideas behind what they came together for.

Another factor behind this purely financial centric trend is in the fact that investors make far more than inventors do. It is not profitable to invent. The reality is that inventors have to sell if they wish to see their work hit the market. This makes for a very direct imbalance of needs and powers.

Inchauste argues that there needs to be movement from just differentiation in the market, to actually making a real difference. Whether that be purely focused on the customer, parts of life, or parts of our world. More powerful goals than just a stake at the ever morphing and inhuman financial system.

\section{Titans}

The discussion shifts toward the larger entities in our digital landscape. There has been a massive shift of big money as the information age took over the industrial age. Yet when the major pivots of the information age are scrutinized it can be seen that jobs are on a decline (lower employee numbers.) Inchauste notes that this age has great benefit for the highly skilled worker and absolutely no room for the average person. As the moving economic balances continue to adjust the workplace landscape, this imbalance will become more and more apparent. 

\newthought{The argument connects cleanly} to the previously mentioned concept of the Technium. Technology that serves as the bread and butter of both the startups and the Goliaths of the industry has become less and less transparent. Not in the closed source sense, but rather in that the degree of complexity required to understand a \textit{complete system} is far beyond the scope of an individual or even a small group.

\section{Minimum Viable Product}

In order to create rapid financial success startups have looked to the idea of the Minimum Viable Product. This model brings forth the quite deadly startup trap mentioned earlier. Though, perhaps more importantly it established foundations today that see profit as the goal, rather than only one of many results to strive for. A minimum viable product treats users as collateral damage on the way to a better product, and faster turnaround for the company.

\newthought{Inchauste lists} purpose, belief, moderation, human goals, and virtuousness as the keys to moving away from the shaky foundations that are being built on top of only cash-flow today. He brings up `wicked problems', how they need to be tackled and approached without profit being the only goal.

\section{Minimum Meaningful Product}

Minimum meaningful product is the proposed change in the direction of startup culture. Inchauste argues that it should:
\begin{description}
  \item[Chose better problems] \hfill \\ Rather than simply ones that guarantee to make a profit.
  \item[Improve People's Lives] \hfill \\ In a meaningful to them way.
  \item[Solve blindness to tedious unpleasant tasks] \hfill \\ Rather than providing a marginal improvement over an existing solution.
\end{description}

With this strategy companies should add the goal of creating meaningful and lasting connections with their customers at the forefront. Products need to be different in a way that is meaningful to their users, not merely differentiated from whatever happens to be the possible competition at the time.

Projects that focus on being meaningful as a major goal are far more sustainable than if they only look toward income. 

Inchauste encourages to think further into the future, to consider where the products and services we create now really fit in over time.

\nocite{Inchauste}
\bibliographystyle{plain}
\bibliography{resources}

\end{document}