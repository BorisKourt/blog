\documentclass{scrartcl}
\usepackage[T1]{fontenc}
\usepackage{baskervald}
\usepackage{graphicx}
\usepackage{datetime}
\usepackage[]{minted}
\usepackage{tcolorbox}
\usepackage{etoolbox}
\BeforeBeginEnvironment{minted}{\begin{tcolorbox}[arc=0mm,colback=yellow!9!white,colframe=yellow!9!white]}
\AfterEndEnvironment{minted}{\end{tcolorbox}}
\title{Scenario Planning Exercise}
\author{Boris Kourtoukov}
\date{Feb 23rd 2014}
\begin{document}
\maketitle
\subsection{Details}
\begin{description}
  \item[Date] Feb 24th 2014
  \item[Time] 1:00pm
  \item[Duration] 40 Minutes
  \item[Who] Leadership in the digital economy class
  \item[Facilitators] Boris Kourtoukov
\end{description}
\subsection{Goals} 
\begin{enumerate}
	\item Engage the class in the ideas proposed by Inchauste's essay.
	\item Build on the subject matter by projecting scenarios 20 years into the future.
	\item Assess the attitudes toward the future of innovation and startups.
\end{enumerate}
\subsection{Agenda} 
\begin{description}
  \item[0 - 10] Describe the arguments and provide four quadrants to chose from
  \item[10 - 25] Allow groups to build the future scenarios based on two possible futures (quadrants).
  \item[25 - 40] Discuss group outcomes.
\end{description}
\subsection{Inputs}
The primary mode of communication will be the blackboard which will be used to illustrate the four quadrants that define future scenarios as proposed by Francisco Inchauste's essay.
\subsection{Outputs} 
The output of this facilitation exercise will be the possibilities within the projected future scenarios. There should be some clear negative connotation based on lack of innovation and hesitation for change. While also some clear indication of some possible futures containing a high degree of forward thinking developments.
\subsection{Key questions} 
Talk about the likelihood of the various futures representing ours, where do participants feel the quadrants overlap the most.
\subsection{Next steps after workshop}
Construct a response based on the experience of running the workshop and the ideas that come up during the exercise.
\end{document}