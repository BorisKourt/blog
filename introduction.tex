\documentclass{scrartcl}
\usepackage[T1]{fontenc}
\usepackage{baskervald}
\usepackage{graphicx}
\usepackage{datetime}
\usepackage[]{minted}
\usepackage{tcolorbox}
\usepackage{etoolbox}
\BeforeBeginEnvironment{minted}{\begin{tcolorbox}[arc=0mm,colback=yellow!9!white,colframe=yellow!9!white]}
\AfterEndEnvironment{minted}{\end{tcolorbox}}
\newcommand{\latestKnownVersion}[1]{v0-01}
\title{Restarting in LaTeX}
\author{Boris Kourtoukov}
\date{modified on: \today}
\begin{document}
\maketitle

I have decided to convert my blog into \LaTeX. Largely to see where this can go but also to have more control over my content. (see figure~\ref{fig:marginfig})

The Tufte-LATEX document classes define a style similar to the style
Edward Tufte uses in his books and handouts. Tufte's style is known
for its extensive use of sidenotes, tight integration of graphics with
text, and well-set typography. This document aims to be at once a
demonstration of the features of the Tufte-LATEX document classes
and a style guide to their use.

\begin{marginfigure}%
  \includegraphics[width=\linewidth]{helix}
  \caption{This is a margin figure .  The helix is defined by 
    $x = \cos(2\pi z)$, $y = \sin(2\pi z)$, and $z = [0, 2.7]$.  The figure was
    drawn using Asymptote (\url{http://asymptote.sf.net/}).}
  \label{fig:marginfig}
\end{marginfigure}

Each post is now a dated .pdf \footnote{This is a sidenote that was entered
using the \texttt{\textbackslash footnote} command.} with it's .tex source available for browsing. All content should remain as is (relatively) but I am not aiming for a lot of backwards compatibility.  

The Tufte-LATEX document classes define a style similar to the style
Edward Tufte uses in his books and handouts. Tufte's style is known
for its extensive use of sidenotes, tight integration of graphics with
text, and well-set typography. This document aims to be at once a
demonstration of the features of the Tufte-LATEX document classes
and a style guide to their use.

\begin{table}[ht]
  \centering
  \fontfamily{ppl}\selectfont
  \begin{tabular}{ll}
    \toprule
    Margin & Length \\
    \midrule
    Paper width & \unit[8\nicefrac{1}{2}]{inches} \\
    Paper height & \unit[11]{inches} \\
    Textblock width & \unit[6\nicefrac{1}{2}]{inches} \\
    Textblock/sidenote gutter & \unit[\nicefrac{3}{8}]{inches} \\
    Sidenote width & \unit[2]{inches} \\
    \bottomrule
  \end{tabular}
  \caption{Here are the dimensions of the various margins used in the Tufte-handout class.}
  \label{tab:normaltab}
\end{table}

The Tufte-LATEX document classes define a style similar to the style
Edward Tufte uses in his books and handouts. Tufte's style is known
for its extensive use of sidenotes, tight integration of graphics with
text, and well-set typography. This document aims to be at once a
demonstration of the features of the Tufte-LATEX document classes
and a style guide to their use.

The Tufte-LATEX document classes define a style similar to the style
Edward Tufte uses in his books and handouts. Tufte's style is known
for its extensive use of sidenotes, tight integration of graphics with
text, and well-set typography. This document aims to be at once a
demonstration of the features of the Tufte-LATEX document classes
and a style guide to their use.

\newthought{In his later books} Tufte
starts each section with a bit of vertical space, a non-indented paragraph,
and sets the first few words of the sentence in \textsc{small caps}.  To
accomplish this using this style, use the \Verb|\newthought| command

The Tufte-LATEX document classes define a style similar to the style
Edward Tufte uses in his books and handouts. Tufte's style is known
for its extensive use of sidenotes, tight integration of graphics with
text, and well-set typography. This document aims to be at once a
demonstration of the features of the Tufte-LATEX document classes
and a style guide to their use.

\vspace{1in}
\section{Endcap}
For the latest updates Twitter (\url{https://twitter.com/boriskourt}) and the GitHub repository (\url{https://github.com/boriskourt/blog}) are great places to start.
\begin{flushright}
The latest know version of this file is \emph{\latestKnownVersion}
\end{flushright}
\end{document}